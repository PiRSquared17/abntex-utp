%% vim:ft=tex
\documentclass[font=plain,chapter=TITLE,section=Title,espaco=duplo,tocpage=plain,appendix=Name,floatnumber=continuous]{abnt}
\usepackage{hyperref}
\usepackage[utf8]{inputenc}
\usepackage[brazil]{babel}
\usepackage[alf]{abntcite}
\usepackage{graphicx}

%% xunxo para seguir as normas da UTP
\usepackage{xunxos-utp}

%% informações sobre o trabalho
\autor{Nicolai Nicolaiev}
\coautor{Fulano Fulanowsky}
\titulo{Um exemplo de trabalho em \LaTeX{}}
\comentario{Trabalho de Conclusão de Curso apresentado ao Curso de Bacharelado
em Ciência da Computação, da Faculdade de Ciências Exatas da Universidade
Tuiuti do Paraná, como requisito parcial para a obtenção do grau de Bacharel em
Ciência da Computação.}
\instituicao{Universidade Tuiuti do Paraná}
\orientador[Orientador: ]{Parararan Parararanavam}
\local{Curitiba}
\data{2010}

\begin{document}
%% xunxos-utp.sty
%% Copyright 2010 Bogdano Arendartchuk <bhdn@ukr.net>
%
% This work may be distributed and/or modified under the
% conditions of the LaTeX Project Public License, either version 1.3
% of this license or (at your option) any later version.
% The latest version of this license is in
%   http://www.latex-project.org/lppl.txt
% and version 1.3 or later is part of all distributions of LaTeX
% version 2005/12/01 or later.
%
% This work has the LPPL maintenance status `maintained'.
% 
% The Current Maintainer of this work is Bogdano Arendartchuk.
%
% This work consists of the files xunxos-utp.sty and xunxos-utp-doc.tex.
%
\renewcommand{\figurename}{FIGURA}
\renewcommand{\tablename}{TABELA}
\renewcommand{\apudname}{\emph{apud}}
\counterwithout{footnote}{chapter}
\counterwithout{equation}{chapter}
\counterwithout{figure}{chapter}
\citeoption{minhasopcoes}
\nohyphens


\UTPCapa
\UTPFalsaFolhaDeRosto
\UTPFolhaDeRosto

\begin{resumo}
TODO: Resumo.

Palavras-chave: TODO; TODO
\end{resumo}

\listoffigures
\listoftables
\listadequadros
\sumario

\chapter{Introdução}

TODO: Introdução

\chapter{Revisão da Literatura}

TODO: Revisão da literatura.

\chapter{Metodologia de desenvolvimento}

TODO: Metodologia.

\chapter{Conclusão}

TODO: Conclusão.

\chapter{Considerações Finais}

TODO: Considerações Finais.

%
% As entradas de bibliográficas (as "referências") estão em biblio.bib.
%
\bibliography{biblio}

\end{document}
